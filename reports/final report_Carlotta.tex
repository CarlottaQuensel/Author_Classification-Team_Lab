\pdfoutput=1
\documentclass[11pt]{article}
\usepackage{acl}
\usepackage{times}
\usepackage{latexsym}
\usepackage[T1]{fontenc}
\usepackage[utf8]{inputenc}
\usepackage{microtype}

\title{Author Prediction for Poetry}

\author{Katrin Schmidt \\
   Immatriculation-no\\
  \texttt{@ims.uni-stuttgart.de} \\\And
  Carlotta Quensel \\
  3546286 \\
  \texttt{@ims.uni-stuttgart.de} \\}

\begin{document}
\maketitle
\begin{abstract}
Same structure as the whole paper, but in short
\end{abstract}

\section{Introduction}
Short motivation and explanation of relevancy
of your task, research questions/hypothesis
\begin{itemize}
\item what is author classification
\item why poetry
\item research question
\begin{itemize}
\item Goal: find features inherent to poetry\\- is our goal possible\\- which features are good
\item Problem: style features might depend on medium (e.g. Limmerick) more than on the author
\end{itemize}
\item Motivation: ??
\end{itemize}


\section{Method}

Description of your method (e.g. perceptron) without talking about the specific task too much. Explain features used (with or without being task specific), but do not judge them.
\begin{itemize}
\item Maximum Entropy classifier (explain why not other approaches)\\
      - short program description (training, classification)
\item Features: MaxEnt/Bag-of-Words(/poetry specific?), learnFeatures (PMI)
\item subsection with data/corpus creation from Poetry Foundation\\
      - which information is included\\
      - preprocessing steps (tokenizing)\\
      - statistics (number of poems/poets/poems per poet with graph)
\end{itemize}

\section{Experiments}

\subsection{Experimental Design}

Explain how you perform your experiments, which data is used, statistics of data.
\begin{itemize}
\item data statistics (decision for number of authors as hyperparameter)
\item Train/Test split
\item Program:\\
- Hyperparameters: accuracy threshold, track loss \& accuracy, \#{}features/author
\item Baseline (bag of words), Advanced: \#{}verses, \#{}stanzas, rhyme scheme
\end{itemize}

\subsection{Results}
Explain how your model performs, different models or configurations of your models.
\begin{itemize}
\item Feature combinations: baseline=BoW, advanced=all, other=?
\item recall/precision/f$_1$ for all combinations (table)
\end{itemize}

\subsection{Error Analysis}

Given the configurations in the Results section, what are frequent sources of errors
\begin{itemize}
\item specifics and numbers about errors?
\item overprediction of alphabetically first author
\item many authors not predicted (uneven data distribution or bad features)
\item feature weights converge similarly (no real weighting)
\end{itemize}

\section{Summary \& Conclusion}

Explain and summarize your results on a more abstract level. What is good, what is not so
good. What are the main contributions in your experiments?


\section{Future Work}

What did you have in mind what else your
would have liked to experiment with? Other ideas?
\begin{itemize}
\item other models (e.g. Neural Net)
\item other features (Topics from Poetry Foundation website)
\item feature interdependencies/more data analysis
\item genre interaction with author classification (multitask learning?)
\end{itemize}


% custom entries
\bibliographystyle{acl_natbib}
\bibliography{custom.bib}

\appendix

\section{Contributions}\label{sec:cont}
Who implemented what?
Who participated in the design of which components?
Who wrote which part of the review?
\section{Declaration of Originality}
we hereby certify that this report has been composed by us and is based on our own work, unless 
stated otherwise. No other person’s work has been used without due acknowledgement our own contributions 
are listed under \ref{sec:cont}. All references and verbatim extracts have been quoted, and all 
sources of information, including graphs and data sets, have been specifically acknowledged.

\end{document}
