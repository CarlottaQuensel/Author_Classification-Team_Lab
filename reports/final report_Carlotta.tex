\pdfoutput=1
\documentclass[11pt]{article}
\usepackage{acl}
\usepackage{times}
\usepackage{latexsym}
\usepackage[T1]{fontenc}
\usepackage[utf8]{inputenc}
\usepackage{microtype}

\title{Author Prediction for Poetry}

\author{Katrin Schmidt \\
   Immatriculation-no\\
  \texttt{@ims.uni-stuttgart.de} \\\And
  Carlotta Quensel \\
  3546286 \\
  \texttt{@ims.uni-stuttgart.de} \\}

\begin{document}
\maketitle
\begin{abstract}
Same structure as the whole paper, but in short
\end{abstract}

\section{Introduction}
Short motivation and explanation of relevancy
of your task, research questions/hypothesis
\begin{itemize}
\item what is author classification
\item why poetry
\item research question
\begin{itemize}
\item Goal: find features inherent to poetry\\- is our goal possible\\- which features are good
\item Problem: style features might depend on medium (e.g. Limmerick) more than on the author
\end{itemize}
\item Motivation: ??
\end{itemize}


\section{Method}

Description of your method (e.g. perceptron) without talking about the specific task too much. Explain features used (with or without being task specific), but do not judge them.
\begin{itemize}
\item Maximum Entropy classifier (explain why not other approaches)\\
      - short program description (training, classification)
\item Features: MaxEnt/Bag-of-Words(/poetry specific?), learnFeatures (PMI)
\item subsection with data/corpus creation from Poetry Foundation\\
      - which information is included\\
      - preprocessing steps (tokenizing)\\
      - statistics (number of poems/poets/poems per poet with graph)
\end{itemize}

\label{sec:corpus}

\section{Experiments}

\subsection{Experimental Design}

For all model configurations, our hyperparameters of the number of authors, the maximum training 
iterations, the accuracy threshold and the number of features per author were left untouched. The training stopped when the accuracy
improvement fell below  0.001 or after 100 iterations (which was never reached during training).

With at least 30 poems per author and 1569 datapoints, the split between test and training data was 
pseudo-randomized 75 to 25 to ensure sufficient coverage of each author in training and evaluation.
This way the least prolific author (Edmund Spenser) had 25 poems for training and eight for 
the evaluation. The poems alloted for training were also used to compute the pointwise mutual information
because of the aforementioned (\ref{sec:corpus}) data sparsity. This was done despite the risk of 
overfitting since a second division of the data would leave us with ten to fifteen poems per author 
for both the feature extraction and training, which is not sufficient for either method (PMI or Maximum
Entropy training).

The features themselves consisted of individual tokens, the number of verses and stanzas as well as the
rhyme scheme for each poem. The word features were obtained by converting the poem into a bag-of-words 
vector and retrieving the value (0 or 1) for a specific word in the vocabulary. This vocabulary was built
from the tokenized training data as the classifier will only learn weights for features that are seen during
training. A simple classifier with only the tokens was trained as a baseline for the poetry specific features.

The rhyme scheme was obtained from the first four lines of the poems. While there might be rhymes that span more
than four lines (\textit{\textbf{a}bcd\textbf{a}}), this is highly unlikely without a repetition of the first 
rhyme or another rhyme pair in the lines in between 1 and 5. The scheme was constructed by consecutively taking 
the last word of the first unmatched line and checking all other unmatched lines for rhymes with 
\texttt{pronouncing.rhymes(word)} (\textcolor{red}{Parrish, 2015}). This lead to a four letter string for each poem of the form:
\[``a\{a,b\}\{a,b,c\},\{a,b,c,d\}"\]
The number of verses per poem were sorted in \textcolor{red}{x-line steps} after looking at the distribution
in the training data. The steps were converted into bins of  at least and at most $x$ number of verses. 
Similarly, the number of blank lines in a poem was used to determine the number of stanzas and sorted 
into steps of \textcolor{red}{x, y or z} stanzas. 

For our first experiments, the classifier was initialized with the 30 most informative features per author, which for
30 authors resulted in 900 features whose weights were trained. We compared the baseline of just words to a classifier 
with all features ("full"), combinations of words and only one of the advanced features (from here on referred to by the name
of that feature, i.e. stanza model, rhyme model, verse model) and a model with all features ecxept for the words.
After comparing these models with fixed hyperparameters, we changed the number of learned features and the size of the 
author set to observe the parameters' effect.

\subsection{Results}
For the training, the models rarely performed more than ten optimization steps before the accuracy stopped changing. 
Tracking the aggregated loss as well as the accuracy showed us that the loss was still high when the accuracy stopped
improving. \textcolor{red}{here graph?} Training of the baseline model terminated with an accuracy \textcolor{red}{50\%} on the training data,
which dropped to a \textcolor{red}{micro} f$_1$ score of .115 with the unseen test data. This pattern was also present in the other model
configurations as shown in table \ref{tab:eval}.

\begin{table}
      \begin{tabular}{lccccc}\hline
            & Baseline & Full & Verse & Stanza & Rhyme\\\hline
            Accuracy & \\
            Precision & \\
            Recall & \\
            micro F$_1$ & \\
            macro F$_1$ & &&&&\\\hline
      \end{tabular}
      \caption{Evaluation of the different model configurations on the unseen test data and the model's
      accuracy on the training data for comparison.}\label{tab:eval}
\end{table}

\subsection{Error Analysis}

Given the configurations in the Results section, what are frequent sources of errors
\begin{itemize}
\item specifics and numbers about errors?
\item overprediction of alphabetically first author
\item many authors not predicted (uneven data distribution or bad features)
\item feature weights converge similarly (no real weighting)
\end{itemize}

\section{Summary \& Conclusion}

Explain and summarize your results on a more abstract level. What is good, what is not so
good. What are the main contributions in your experiments?


\section{Future Work}

What did you have in mind what else your
would have liked to experiment with? Other ideas?
\begin{itemize}
\item other models (e.g. Neural Net)
\item other features (Topics from Poetry Foundation website)
\item feature interdependencies/more data analysis
\item genre interaction with author classification (multitask learning?)
\end{itemize}


% custom entries
\bibliographystyle{acl_natbib}
\bibliography{custom.bib}

\appendix

\section{Contributions}\label{sec:cont}
Who implemented what?
Who participated in the design of which components?
Who wrote which part of the review?
\section{Declaration of Originality}
we hereby certify that this report has been composed by us and is based on our own work, unless 
stated otherwise. No other person’s work has been used without due acknowledgement our own contributions 
are listed under \ref{sec:cont}. All references and verbatim extracts have been quoted, and all 
sources of information, including graphs and data sets, have been specifically acknowledged.

\end{document}
