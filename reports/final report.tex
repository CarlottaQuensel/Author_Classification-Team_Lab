\pdfoutput=1
\documentclass[11pt]{article}
\usepackage{acl}
\usepackage{natbib}
\usepackage{times}
\usepackage{latexsym}
\usepackage[T1]{fontenc}
\usepackage[utf8]{inputenc}
\usepackage{microtype}
\usepackage{amsmath, amsthm, amssymb, amsfonts}
\usepackage{graphicx}

\title{Author Prediction for Poetry}

\author{Katrin Schmidt \\
   3197512\\
  \texttt{@ims.uni-stuttgart.de} \\\And
  Carlotta Quensel \\
  3546286 \\
  \texttt{@ims.uni-stuttgart.de} \\}

\begin{document}
\maketitle
\begin{abstract}
Same structure as the whole paper, but in short
\end{abstract}

\section{Introduction}
The task of author classification refers to the task of predicting the author of a given text by analyzing the text in terms of some features \citet{Argamon:2009}. These features can then be compared with those learned from some training data. A common approach to author classification "[...] consists of a combination of text analysis for extracting document features that are predictive of the author, and text categorization using Machine Learning (ML) techniques" \citet{Luyckx:2011}. For this paper we focus on the prediction of poems in order to examine which features are particularly useful for the author classification of poems. This will give information about linguistical aspects on the one hand, that is poem specific features, and computational aspects on the other hand, that is the the question of efficiency and performance of the implementation of these features. Thus the goal is to answer the question if there exist features that are inherent to poetry and then to analyse and judge them according to their performance.\\
Besides from that we will take into account that style features might depend on medium (e.g. Limmerick) more than on the author.


\subsection{Corpus Creation}
As a training data we used the collection of the Poetry Foundation which is pulled from kaggle.com as a premade csv-database. The dataset consists of 15 567 poems, written by altogether 3 309 authors. The following graphic shows the distribution of poems per author. \\

\includegraphics[width=0.45\textwidth]{/Users/katrin/Desktop/Master/Team_Lab/Report/Statistics.png}
\\

Since the data includes many authors who wrote between 1 and 5 poems, we used only the 30 most prolific authors to get enough data points per class for the method. Finally we ended up with 1 569 poems which is barely 10 \% from the original dataset.
In order to train our model with the data, we sorted the poems by author, normalized some remaining unicode strings (e.g. "ax0", which represents whitespaces) and tokenized them with the NLTK WordPunctTokenizer. Then we split the data into train and test set and converted the poems into bag-of-word vectors using the vocabulary in the train set.

\section{Method}
Generally speaking, a maximum entropy classifier is used for generating a probability distribution based on some training data \citet{Nigam:1999}. Before the classifier begins to train, the probabilities should be equally distributed, since there is no bias towards any label - therefore, the entropy is maximal in the beginning, in other words, the weights associated with the features are unknown \citet{Nigam:1999}. The formula for the maximum entropy classifier is given by 
\[p_{\lambda}(y|\pmb{x}) = \frac{\text{exp} \sum_i \lambda_i f_i(y,\pmb{x})}{\sum_{y'} \text{exp} \sum_i \lambda_i f_i (y',\pmb{x})}\]
where $f_i(y, \pmb{x})$ is a feature and $\lambda_i$ the corresponding weight \citet{Nigam:1999}.
Furthermore, the maximum entropy classifier presupposes a dependence relation between the features \citet{Osborne:2002}. This means that the classifier is not only able to differentiate between features that are relevant and features that are irrelevant for the classification task, but also to include this information in its classification process \citet{Osborne:2002}.
We decided to choose this classifier since we assume that features which match a certain author are relevant whereas features that don't match the author are irrelevant.

For our classification task a feature $f_{i}$ contains a data property paired with a label, where $\pmb{x}$ is a document vector and $y$ is a label, so that

\[f_{i}(y,\pmb{x})=\begin{cases}
1& \text{if property of }\pmb{x} \text{ occurs with label y}\\
0& \text{otherwise.} \\
\end{cases}\] 
More precise the document vector is stored as bag-of-words vector. The feature is 1 if the property occurs together with the label and 0 if not.
The features are learned from data with pointwise mutual information (PMI), which is obtained by

\[\text{PMI}(x,y)=\text{lg}\frac{P(x,y)}{P(x)P(y)}\]
and can be understood as an association measure that helps to decide whether a feature is informative or not \citet{Bouma:2009}.This ensures that only relevant features are considered \citet{Bouma:2009}. By doing so, the classification process works faster and returns more reliable results.
After learning the features, the classifier assumes some random weights between -10 and +10 for each feature and enters an iterative process to improve the feature weights. The iterative training of the weights is done by calculating the derivation of the weights and adding them to the current weights. Then it checks if the accuracy has been improved. 


\section{Experiments}

\subsection{Experimental Design}

For all model configurations, our hyperparameters of the number of authors, the maximum training 
iterations, the accuracy threshold and the number of features per author were left untouched. The training stopped when the accuracy
improvement fell below  0.001 or after 100 iterations (which was never reached during training).

With at least 30 poems per author and 1569 datapoints, the split between test and training data was 
pseudo-randomized 75 to 25 to ensure sufficient coverage of each author in training and evaluation.
This way the least prolific author (Edmund Spenser) had 25 poems for training and eight for 
the evaluation. The poems alloted for training were also used to compute the pointwise mutual information
because of the aforementioned (\ref{sec:corpus}) data sparsity. This was done despite the risk of 
overfitting since a second division of the data would leave us with ten to fifteen poems per author 
for both the feature extraction and training, which is not sufficient for either method (PMI or Maximum
Entropy training).

The features themselves consisted of individual tokens, the number of verses and stanzas as well as the
rhyme scheme for each poem. The word features were obtained by converting the poem into a bag-of-words 
vector and retrieving the value (0 or 1) for a specific word in the vocabulary. This vocabulary was built
from the tokenized training data as the classifier will only learn weights for features that are seen during
training. A simple classifier with only the tokens was trained as a baseline for the poetry specific features.

The rhyme scheme was obtained from the first four lines of the poems. While there might be rhymes that span more
than four lines (\textit{\textbf{a}bcd\textbf{a}}), this is highly unlikely without a repetition of the first 
rhyme or another rhyme pair in the lines in between 1 and 5. The scheme was constructed by consecutively taking 
the last word of the first unmatched line and checking all other unmatched lines for rhymes with 
\texttt{pronouncing.rhymes(word)} (\textcolor{red}{Parrish, 2015}). This lead to a four letter string for each poem of the form:
\[``a\{a,b\}\{a,b,c\},\{a,b,c,d\}"\]
The number of verses per poem were sorted in \textcolor{red}{x-line steps} after looking at the distribution
in the training data. The steps were converted into bins of  at least and at most $x$ number of verses. 
Similarly, the number of blank lines in a poem was used to determine the number of stanzas and sorted 
into steps of \textcolor{red}{x, y or z} stanzas. 

For our first experiments, the classifier was initialized with the 30 most informative features per author, which for
30 authors resulted in 900 features whose weights were trained. We compared the baseline of just words to a classifier 
with all features ("full"), combinations of words and only one of the advanced features (from here on referred to by the name
of that feature, i.e. stanza model, rhyme model, verse model) and a model with all features ecxept for the words.
After comparing these models with fixed hyperparameters, we changed the number of learned features and the size of the 
author set to observe the parameters' effect.

\subsection{Results}
For the training, the models rarely performed more than ten optimization steps before the accuracy stopped changing. 
Tracking the aggregated loss as well as the accuracy showed us that the loss was still high when the accuracy stopped
improving. \textcolor{red}{here graph?} Training of the baseline model terminated with an accuracy \textcolor{red}{50\%} on the training data,
which dropped to a \textcolor{red}{micro} f$_1$ score of .115 with the unseen test data. This pattern was also present in the other model
configurations as shown in table \ref{tab:eval}.

\begin{table}
      \begin{tabular}{lccccc}\hline
            & Baseline & Full & Verse & Stanza & Rhyme\\\hline
            Accuracy & \\
            Precision & \\
            Recall & \\
            micro F$_1$ & \\
            macro F$_1$ & &&&&\\\hline
      \end{tabular}
      \caption{Evaluation of the different model configurations on the unseen test data and the model's
      accuracy on the training data for comparison.}\label{tab:eval}
\end{table}


\subsection{Error Analysis}

Given the configurations in the Results section, what are frequent sources of errors
\begin{itemize}
\item specifics and numbers about errors?
\item overprediction of alphabetically first author
\item many authors not predicted (uneven data distribution or bad features)
\item feature weights converge similarly (no real weighting)
\end{itemize}

\section{Summary \& Conclusion}

Explain and summarize your results on a more abstract level. What is good, what is not so
good. What are the main contributions in your experiments?


\section{Future Work}

What did you have in mind what else your
would have liked to experiment with? Other ideas?
\begin{itemize}
\item other models (e.g. Neural Net)
\item other features (Topics from Poetry Foundation website)
\item feature interdependencies/more data analysis
\item genre interaction with author classification (multitask learning?)
\end{itemize}


\bibliographystyle{acl_natbib}
\bibliography{custom}



\appendix

\section{Contributions}
Who implemented what?
Who participated in the design of which components?
Who wrote which part of the review?
\section{Declaration of Originality}
\label{sec:appendix}

\end{document}
