\pdfoutput=1
\documentclass[11pt]{article}
\usepackage{acl}
\usepackage{times}
\usepackage{latexsym}
\usepackage[T1]{fontenc}
\usepackage[utf8]{inputenc}
\usepackage{microtype}

\title{Author Prediction for Poetry}

\author{Katrin Schmidt \\
   Immatriculation-no\\
  \texttt{@ims.uni-stuttgart.de} \\\And
  Carlotta Quensel \\
  3546286 \\
  \texttt{@ims.uni-stuttgart.de} \\}

\begin{document}
\maketitle
\begin{abstract}
Same structure as the whole paper, but in short
\end{abstract}

\section{Introduction}

Short motivation and explanation of relevancy
of your task, research questions/hypothesis


\section{Method}

Description of your method (e.g. perceptron) without talking about the specific task too much. Explain features used (with or without being task specific), but do not judge them.

\section{Experiments}

\subsection{Experimental Design}

Explain how you perform your experiments, which data is used, statistics of data.

\subsection{Results}

Explain how your model performs, different models or configurations of your models.

\subsection{Error Analysis}

Given the configurations in the Results section, what are frequent sources of errors

\section{Summary \& Conclusion}

Explain and summarize your
results on a more abstract level. What is good, what is not so
good. What are the main contributions in your experiments?


\section{Future Work}

What did you have in mind what else your
would have liked to experiment with? Other ideas?


% custom entries
\bibliographystyle{acl_natbib}
\bibliography{custom.bib}

\appendix

\section{Contributions}
Who implemented what?
Who participated in the design of which components?
Who wrote which part of the review?
\section{Declaration of Originality}
\label{sec:appendix}

\end{document}
